\documentclass[journal,12pt,twocolumn]{IEEEtran}

\usepackage{setspace}
\usepackage{gensymb} 

\singlespacing


\usepackage[cmex10]{amsmath}

\usepackage{amsthm}


\usepackage{mathrsfs}
\usepackage{txfonts}
\usepackage{stfloats}
\usepackage{bm}
\usepackage{cite}
\usepackage{cases}
\usepackage{subfig}

\usepackage{longtable}
\usepackage{multirow}

\usepackage{enumitem}
\usepackage{mathtools}
\usepackage{steinmetz}
\usepackage{tikz}
\usepackage{circuitikz}
\usepackage{verbatim}
\usepackage{tfrupee}
\usepackage[breaklinks=true]{hyperref}
\usepackage{graphicx}
\usepackage{tkz-euclide}

\usetikzlibrary{calc,math}
\usepackage{listings}
    \usepackage{color}                                            %%
    \usepackage{array}                                            %%
    \usepackage{longtable}                                        %%
    \usepackage{calc}                                             %%
    \usepackage{multirow}                                         %%
    \usepackage{hhline}                                           %%
    \usepackage{ifthen}                                           %%
    \usepackage{lscape}     
\usepackage{multicol}
\usepackage{chngcntr}

\DeclareMathOperator*{\Res}{Res}

\renewcommand\thesection{\arabic{section}}
\renewcommand\thesubsection{\thesection.\arabic{subsection}}
\renewcommand\thesubsubsection{\thesubsection.\arabic{subsubsection}}

\renewcommand\thesectiondis{\arabic{section}}
\renewcommand\thesubsectiondis{\thesectiondis.\arabic{subsection}}
\renewcommand\thesubsubsectiondis{\thesubsectiondis.\arabic{subsubsection}}


\hyphenation{op-tical net-works semi-conduc-tor}
\def\inputGnumericTable{}                                 %%

\lstset{
%language=C,
frame=single, 
breaklines=true,
columns=fullflexible
}
\begin{document}


\newtheorem{theorem}{Theorem}[section]
\newtheorem{problem}{Problem}
\newtheorem{proposition}{Proposition}[section]
\newtheorem{lemma}{Lemma}[section]
\newtheorem{corollary}[theorem]{Corollary}
\newtheorem{example}{Example}[section]
\newtheorem{definition}[problem]{Definition}

\newcommand{\BEQA}{\begin{eqnarray}}
\newcommand{\EEQA}{\end{eqnarray}}
\newcommand{\define}{\stackrel{\triangle}{=}}
\bibliographystyle{IEEEtran}

\providecommand{\mbf}{\mathbf}
\providecommand{\pr}[1]{\ensuremath{\Pr\left(#1\right)}}
\providecommand{\qfunc}[1]{\ensuremath{Q\left(#1\right)}}
\providecommand{\sbrak}[1]{\ensuremath{{}\left[#1\right]}}
\providecommand{\lsbrak}[1]{\ensuremath{{}\left[#1\right.}}
\providecommand{\rsbrak}[1]{\ensuremath{{}\left.#1\right]}}
\providecommand{\brak}[1]{\ensuremath{\left(#1\right)}}
\providecommand{\lbrak}[1]{\ensuremath{\left(#1\right.}}
\providecommand{\rbrak}[1]{\ensuremath{\left.#1\right)}}
\providecommand{\cbrak}[1]{\ensuremath{\left\{#1\right\}}}
\providecommand{\lcbrak}[1]{\ensuremath{\left\{#1\right.}}
\providecommand{\rcbrak}[1]{\ensuremath{\left.#1\right\}}}
\theoremstyle{remark}
\newtheorem{rem}{Remark}
\newcommand{\sgn}{\mathop{\mathrm{sgn}}}
\providecommand{\abs}[1]{\left\vert#1\right\vert}
\providecommand{\res}[1]{\Res\displaylimits_{#1}} 
\providecommand{\norm}[1]{\left\lVert#1\right\rVert}
%\providecommand{\norm}[1]{\lVert#1\rVert}
\providecommand{\mtx}[1]{\mathbf{#1}}
\providecommand{\mean}[1]{E\left[ #1 \right]}
\providecommand{\fourier}{\overset{\mathcal{F}}{ \rightleftharpoons}}
%\providecommand{\hilbert}{\overset{\mathcal{H}}{ \rightleftharpoons}}
\providecommand{\system}{\overset{\mathcal{H}}{ \longleftrightarrow}}
	%\newcommand{\solution}[2]{\textbf{Solution:}{#1}}
\newcommand{\solution}{\noindent \textbf{Solution: }}
\newcommand{\cosec}{\,\text{cosec}\,}
\providecommand{\dec}[2]{\ensuremath{\overset{#1}{\underset{#2}{\gtrless}}}}
\newcommand{\myvec}[1]{\ensuremath{\begin{pmatrix}#1\end{pmatrix}}}
\newcommand{\mydet}[1]{\ensuremath{\begin{vmatrix}#1\end{vmatrix}}}

\numberwithin{equation}{subsection}

\makeatletter
\@addtoreset{figure}{problem}
\makeatother
\let\StandardTheFigure\thefigure
\let\vec\mathbf

\renewcommand{\thefigure}{\theproblem}

\def\putbox#1#2#3{\makebox[0in][l]{\makebox[#1][l]{}\raisebox{\baselineskip}[0in][0in]{\raisebox{#2}[0in][0in]{#3}}}}
     \def\rightbox#1{\makebox[0in][r]{#1}}
     \def\centbox#1{\makebox[0in]{#1}}
     \def\topbox#1{\raisebox{-\baselineskip}[0in][0in]{#1}}
     \def\midbox#1{\raisebox{-0.5\baselineskip}[0in][0in]{#1}}
\vspace{3cm}
\title{ASSIGNMENT-1}
\author{C.RAMYA TULASI}

\maketitle
\newpage

\bigskip
\renewcommand{\thefigure}{\theenumi}
\renewcommand{\thetable}{\theenumi}
Download all python codes from 
\begin{lstlisting}
https://github.com/sachinomdubey/Matrix-theory/codes
\end{lstlisting}
%
and latex-tikz codes from 
%
\begin{lstlisting}
https://github.com/sachinomdubey/Matrix-theory
\end{lstlisting}
%
\section{QUESTION NO-2.23}
\item Construct $\triangle LMN$  right angled at $M$ such that
$LN$ = 5 and $MN$ = 3.
%

%
\section{SOLUTION}
 
Let
\begin{align}
\textbf{L}=\myvec{0 \\ l},
\textbf{M}=\myvec{0 \\ 0} ,
\textbf{N}=\myvec{3 \\ 0}
\end{align}
Now,
\begin{align}
\norm{\vec{N}-\vec{M}}^2 = \norm{\vec{N}}^2  = 3^2 =9
\\
\norm{\vec{L}-\vec{M}}^2 = \norm{\vec{L}}^2 =l^2
\end{align}
We know,
\begin{align}
\norm{\vec{L}-\vec{N}}^2 &= ({\vec{L}-\vec{N}})^T({\vec{L}-\vec{N}})
\\
&= \vec{L}^T\vec{L}+\vec{N}^T\vec{N}-\vec{L}^T\vec{N} - \vec{L}^T\vec{L}
\\
&= \norm{\vec{L}}^2 + \norm{\vec{L}}^2 - 2\vec{L}^T\vec{N}
\\
&= \norm{\vec{L}}^2 + \norm{\vec{N}}^2-2.0
\\
&= l^2 + 3^2
\\
&=l^2+9
\end{align}
But we know LN=5 
\begin{align}
SO,
\norm{{\vec{L}-\vec{N}}}^2=5^2=25
\\
l^2+9=25
\\
l^2=25-9
\\
l^2=16
\\
l=4
\end{align}
\begin{align}
Therefore,l=4
\end{align}

Now, Vertices of given $\triangle LMN$ can be written as,
\begin{align}
\vec{L} = \myvec{0\\l}=\myvec{0\\4}, \vec{M} = \myvec{0\\0}, \vec{N} = \myvec{3\\0}
\end{align}

Now,$\triangle LMN$ can be plotted using vertices $LM$ ,$MN$ and $LN$ .
\\
\\
\\


\tikzset{every picture/.style={line width=0.75pt}} %set default line width to 0.75pt        

\begin{tikzpicture}[x=0.75pt,y=0.75pt,yscale=-1,xscale=1]
%uncomment if require: \path (0,300); %set diagram left start at 0, and has height of 300

%Shape: Axis 2D [id:dp5221220009640113] 
\draw  (100,250.4) -- (590,250.4)(149,11) -- (149,277) (583,245.4) -- (590,250.4) -- (583,255.4) (144,18) -- (149,11) -- (154,18) (199,245.4) -- (199,255.4)(249,245.4) -- (249,255.4)(299,245.4) -- (299,255.4)(349,245.4) -- (349,255.4)(399,245.4) -- (399,255.4)(449,245.4) -- (449,255.4)(499,245.4) -- (499,255.4)(549,245.4) -- (549,255.4)(144,200.4) -- (154,200.4)(144,150.4) -- (154,150.4)(144,100.4) -- (154,100.4)(144,50.4) -- (154,50.4) ;
\draw   (206,262.4) node[anchor=east, scale=0.75]{1} (256,262.4) node[anchor=east, scale=0.75]{2} (306,262.4) node[anchor=east, scale=0.75]{3} (356,262.4) node[anchor=east, scale=0.75]{4} (406,262.4) node[anchor=east, scale=0.75]{5} (456,262.4) node[anchor=east, scale=0.75]{6} (506,262.4) node[anchor=east, scale=0.75]{7} (556,262.4) node[anchor=east, scale=0.75]{8} (146,200.4) node[anchor=east, scale=0.75]{1} (146,150.4) node[anchor=east, scale=0.75]{2} (146,100.4) node[anchor=east, scale=0.75]{3} (146,50.4) node[anchor=east, scale=0.75]{4} ;
%Straight Lines [id:da0249364870780171] 
\draw [color={rgb, 255:red, 80; green, 227; blue, 194 }  ,draw opacity=1 ]   (149,51) -- (149,250.4) ;
%Straight Lines [id:da9133646992394244] 
\draw [color={rgb, 255:red, 208; green, 2; blue, 27 }  ,draw opacity=1 ]   (149,250.4) -- (302,250) ;
%Straight Lines [id:da38380866662909163] 
\draw [color={rgb, 255:red, 74; green, 144; blue, 226 }  ,draw opacity=1 ]   (149,51) -- (297,248) ;
%Shape: Right Angle [id:dp17118155346443342] 
\draw   (150,231) -- (172,231) -- (172,251) ;

% Text Node
\draw (158,31) node [anchor=north west][inner sep=0.75pt]   [align=left] {L};
% Text Node
\draw (132,250) node [anchor=north west][inner sep=0.75pt]   [align=left] {M};
% Text Node
\draw (304,253) node [anchor=north west][inner sep=0.75pt]   [align=left] {N};


\end{tikzpicture}


\end{document}

