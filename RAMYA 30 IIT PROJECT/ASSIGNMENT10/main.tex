\documentclass[journal,12pt,twocolumn]{IEEEtran}

\usepackage{setspace}
\usepackage{gensymb}

\singlespacing


\usepackage[cmex10]{amsmath}

\usepackage{amsthm}

\usepackage{mathrsfs}
\usepackage{txfonts}
\usepackage{stfloats}
\usepackage{bm}
\usepackage{cite}
\usepackage{cases}
\usepackage{subfig}

\usepackage{longtable}
\usepackage{multirow}

\usepackage{enumitem}
\usepackage{mathtools}
\usepackage{steinmetz}
\usepackage{tikz}
\usepackage{circuitikz}
\usepackage{verbatim}
\usepackage{tfrupee}
\usepackage[breaklinks=true]{hyperref}
\usepackage{graphicx}
\usepackage{tkz-euclide}
\usepackage{float}
\usepackage{blkarray}

\usetikzlibrary{calc,math}
\usepackage{listings}
    \usepackage{color}                                            %%
    \usepackage{array}                                            %%
    \usepackage{longtable}                                        %%
    \usepackage{calc}                                             %%
    \usepackage{multirow}                                         %%
    \usepackage{hhline}                                           %%
    \usepackage{ifthen}                                           %%
    \usepackage{lscape}     
\usepackage{multicol}
\usepackage{chngcntr}

\DeclareMathOperator*{\Res}{Res}

\renewcommand\thesection{\arabic{section}}
\renewcommand\thesubsection{\thesection.\arabic{subsection}}
\renewcommand\thesubsubsection{\thesubsection.\arabic{subsubsection}}

\renewcommand\thesectiondis{\arabic{section}}
\renewcommand\thesubsectiondis{\thesectiondis.\arabic{subsection}}
\renewcommand\thesubsubsectiondis{\thesubsectiondis.\arabic{subsubsection}}


\hyphenation{op-tical net-works semi-conduc-tor}
\def\inputGnumericTable{}                                 %%

\lstset{
%language=C,
frame=single, 
breaklines=true,
columns=fullflexible
}
\begin{document}


\newtheorem{theorem}{Theorem}[section]
\newtheorem{problem}{Problem}
\newtheorem{proposition}{Proposition}[section]
\newtheorem{lemma}{Lemma}[section]
\newtheorem{corollary}[theorem]{Corollary}
\newtheorem{example}{Example}[section]
\newtheorem{definition}[problem]{Definition}

\newcommand{\BEQA}{\begin{eqnarray}}
\newcommand{\EEQA}{\end{eqnarray}}
\newcommand{\define}{\stackrel{\triangle}{=}}
\bibliographystyle{IEEEtran}
\providecommand{\mbf}{\mathbf}
\providecommand{\pr}[1]{\ensuremath{\Pr\left(#1\right)}}
\providecommand{\qfunc}[1]{\ensuremath{Q\left(#1\right)}}
\providecommand{\sbrak}[1]{\ensuremath{{}\left[#1\right]}}
\providecommand{\lsbrak}[1]{\ensuremath{{}\left[#1\right.}}
\providecommand{\rsbrak}[1]{\ensuremath{{}\left.#1\right]}}
\providecommand{\brak}[1]{\ensuremath{\left(#1\right)}}
\providecommand{\lbrak}[1]{\ensuremath{\left(#1\right.}}
\providecommand{\rbrak}[1]{\ensuremath{\left.#1\right)}}
\providecommand{\cbrak}[1]{\ensuremath{\left\{#1\right\}}}
\providecommand{\lcbrak}[1]{\ensuremath{\left\{#1\right.}}
\providecommand{\rcbrak}[1]{\ensuremath{\left.#1\right\}}}
\theoremstyle{remark}
\newtheorem{rem}{Remark}
\newcommand{\sgn}{\mathop{\mathrm{sgn}}}
\providecommand{\abs}[1]{\left\vert#1\right\vert}
\providecommand{\res}[1]{\Res\displaylimits_{#1}} 
\providecommand{\norm}[1]{\left\lVert#1\right\rVert}
%\providecommand{\norm}[1]{\lVert#1\rVert}
\providecommand{\mtx}[1]{\mathbf{#1}}
\providecommand{\mean}[1]{E\left[ #1 \right]}
\providecommand{\fourier}{\overset{\mathcal{F}}{ \rightleftharpoons}}
%\providecommand{\hilbert}{\overset{\mathcal{H}}{ \rightleftharpoons}}
\providecommand{\system}{\overset{\mathcal{H}}{ \longleftrightarrow}}
	%\newcommand{\solution}[2]{\textbf{Solution:}{#1}}
\newcommand{\solution}{\noindent \textbf{Solution: }}
\newcommand{\cosec}{\,\text{cosec}\,}
\providecommand{\dec}[2]{\ensuremath{\overset{#1}{\underset{#2}{\gtrless}}}}
\newcommand{\myvec}[1]{\ensuremath{\begin{pmatrix}#1\end{pmatrix}}}
\newcommand{\mydet}[1]{\ensuremath{\begin{vmatrix}#1\end{vmatrix}}}
\numberwithin{equation}{subsection}
\makeatletter
\@addtoreset{figure}{problem}
\makeatother
\let\StandardTheFigure\thefigure
\let\vec\mathbf
\renewcommand{\thefigure}{\theproblem}
\def\putbox#1#2#3{\makebox[0in][l]{\makebox[#1][l]{}\raisebox{\baselineskip}[0in][0in]{\raisebox{#2}[0in][0in]{#3}}}}
     \def\rightbox#1{\makebox[0in][r]{#1}}
     \def\centbox#1{\makebox[0in]{#1}}
     \def\topbox#1{\raisebox{-\baselineskip}[0in][0in]{#1}}
     \def\midbox#1{\raisebox{-0.5\baselineskip}[0in][0in]{#1}}
\vspace{3cm}
\title{Assignment 10}
\author{C.Ramya Tulasi}
\maketitle
\newpage
\bigskip
\renewcommand{\thefigure}{\theenumi}
\renewcommand{\thetable}{\theenumi}
Download all python codes from 
\begin{lstlisting}
https://github.com/CRAMYATULASI/ASSIGNMENT10/tree/main/ASSIGNMENT10/CODES
\end{lstlisting}
%
Latex-tikz codes from 
%
\begin{lstlisting}
https://github.com/CRAMYATULASI/ASSIGNMENT10/tree/main/ASSIGNMENT10
\end{lstlisting}
%
\section{Question No. 2.45}

A manufacturer produces three products x,y,z which he sells in two markets. Annual sales are indicated below:
\begin{table}[!ht]
\centering
    \resizebox{\columnwidth}{9mm}{
\begin{tabular}{|c|c|c|c|}
\hline
Products  & x     & y     & z     \\ 
\hline
Market-I  & 10000 & 2000  & 18000 \\
\hline
Market-II & 6000  & 20000 & 8000  \\ 
\hline
\end{tabular}}
\end{table}

\begin{enumerate}
\item If unit sale prices of x,y and z are \rupee{2.50},\rupee{1.50} and \rupee{1.00} respectively,find the total revenue in each market with the help of matrix algebra.
\item If the unit cost of the above three commodities are \rupee{2.00},\rupee{1.00} and 50 paise respectively.Find the gross profit.
\end{enumerate}
\section{Solution}
Let the sales of the product x,y and z per market be denoted by  matrix A
\begin{align}
\begin{blockarray}{cccc}
\text{x} & \text{y} & \text{z} \\
\begin{block}{(ccc)(c)}
10000 & 2000 & 18000 & \text{Market-I}\\
6000 & 20000 & 8000 & \text{Market-II} \\
\end{block}
\end{blockarray}
\end{align}
\begin {enumerate}
\item
Let the unit sale price of the products x,y and z per market be denoted by matrix B
\begin{align}
\vec{B}=\myvec{2.50\\1.50\\1.00}
\end{align}
Total Revenue in Market-I and Market-II
\begin{align}
\vec{A}\vec{B}&=\myvec{10000&2000&18000\\6000&20000&8000}\myvec{2.50\\1.50\\1.00}\\
&=\myvec{46000\\53000}
\end{align}
\item
Let the unit cost price of the products x,y and z per market be denoted by matrix C
\begin{align}
\vec{C}=\myvec{2.00\\1.00\\0.50}
\end{align}
Total cost of Market-I and Market-II
\begin{align}
\vec{A}\vec{C}&=\myvec{10000&2000&18000\\6000&20000&8000}\myvec{2.00\\1.00\\0.50}\\
&=\myvec{31000\\36000}
\end{align}
$\therefore$ Gross Profit = Total revenue - Total cost
\begin{align}
\vec{A}\vec{B}-\vec{A}\vec{C}&=\myvec{46000\\53000} - \myvec{31000\\36000}  \\
&=\myvec{15000\\17000}
\end{align}
$\therefore$ Total profit in Market-I = 15000

Total profit in Market-II =17000
\end{enumerate}
\numberwithin{figure}{section}
\begin{figure}[!ht]
\centering
\includegraphics[width=\columnwidth]{FIGURE.png}
\caption{Revenue,Sales \& Profit Of Market-I \& II}
\label{fig:Profit}	
\end{figure}
\end{document}
\end{document}
