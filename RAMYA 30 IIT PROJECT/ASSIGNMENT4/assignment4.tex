\documentclass[journal,12pt,twocolumn]{IEEEtran}

\usepackage{setspace}
\usepackage{gensymb}

\singlespacing


\usepackage[cmex10]{amsmath}

\usepackage{amsthm}

\usepackage{mathrsfs}
\usepackage{txfonts}
\usepackage{stfloats}
\usepackage{bm}
\usepackage{cite}
\usepackage{cases}
\usepackage{subfig}

\usepackage{longtable}
\usepackage{multirow}

\usepackage{enumitem}
\usepackage{mathtools}
\usepackage{steinmetz}
\usepackage{tikz}
\usepackage{circuitikz}
\usepackage{verbatim}
\usepackage{tfrupee}
\usepackage[breaklinks=true]{hyperref}
\usepackage{graphicx}
\usepackage{tkz-euclide}
\usepackage{float}

\usetikzlibrary{calc,math}
\usepackage{listings}
    \usepackage{color}                                            %%
    \usepackage{array}                                            %%
    \usepackage{longtable}                                        %%
    \usepackage{calc}                                             %%
    \usepackage{multirow}                                         %%
    \usepackage{hhline}                                           %%
    \usepackage{ifthen}                                           %%
    \usepackage{lscape}     
\usepackage{multicol}
\usepackage{chngcntr}

\DeclareMathOperator*{\Res}{Res}

\renewcommand\thesection{\arabic{section}}
\renewcommand\thesubsection{\thesection.\arabic{subsection}}
\renewcommand\thesubsubsection{\thesubsection.\arabic{subsubsection}}

\renewcommand\thesectiondis{\arabic{section}}
\renewcommand\thesubsectiondis{\thesectiondis.\arabic{subsection}}
\renewcommand\thesubsubsectiondis{\thesubsectiondis.\arabic{subsubsection}}


\hyphenation{op-tical net-works semi-conduc-tor}
\def\inputGnumericTable{}                                 %%

\lstset{
%language=C,
frame=single, 
breaklines=true,
columns=fullflexible
}
\begin{document}
\newtheorem{theorem}{Theorem}[section]
\newtheorem{problem}{Problem}
\newtheorem{proposition}{Proposition}[section]
\newtheorem{lemma}{Lemma}[section]
\newtheorem{corollary}[theorem]{Corollary}
\newtheorem{example}{Example}[section]
\newtheorem{definition}[problem]{Definition}

\newcommand{\BEQA}{\begin{eqnarray}}
\newcommand{\EEQA}{\end{eqnarray}}
\newcommand{\define}{\stackrel{\triangle}{=}}
\bibliographystyle{IEEEtran}
\providecommand{\mbf}{\mathbf}
\providecommand{\pr}[1]{\ensuremath{\Pr\left(#1\right)}}
\providecommand{\qfunc}[1]{\ensuremath{Q\left(#1\right)}}
\providecommand{\sbrak}[1]{\ensuremath{{}\left[#1\right]}}
\providecommand{\lsbrak}[1]{\ensuremath{{}\left[#1\right.}}
\providecommand{\rsbrak}[1]{\ensuremath{{}\left.#1\right]}}
\providecommand{\brak}[1]{\ensuremath{\left(#1\right)}}
\providecommand{\lbrak}[1]{\ensuremath{\left(#1\right.}}
\providecommand{\rbrak}[1]{\ensuremath{\left.#1\right)}}
\providecommand{\cbrak}[1]{\ensuremath{\left\{#1\right\}}}
\providecommand{\lcbrak}[1]{\ensuremath{\left\{#1\right.}}
\providecommand{\rcbrak}[1]{\ensuremath{\left.#1\right\}}}
\theoremstyle{remark}
\newtheorem{rem}{Remark}
\newcommand{\sgn}{\mathop{\mathrm{sgn}}}
\providecommand{\abs}[1]{\left\vert#1\right\vert}
\providecommand{\res}[1]{\Res\displaylimits_{#1}} 
\providecommand{\norm}[1]{\left\lVert#1\right\rVert}
%\providecommand{\norm}[1]{\lVert#1\rVert}
\providecommand{\mtx}[1]{\mathbf{#1}}
\providecommand{\mean}[1]{E\left[ #1 \right]}
\providecommand{\fourier}{\overset{\mathcal{F}}{ \rightleftharpoons}}
%\providecommand{\hilbert}{\overset{\mathcal{H}}{ \rightleftharpoons}}
\providecommand{\system}{\overset{\mathcal{H}}{ \longleftrightarrow}}
	%\newcommand{\solution}[2]{\textbf{Solution:}{#1}}
\newcommand{\solution}{\noindent \textbf{Solution: }}
\newcommand{\cosec}{\,\text{cosec}\,}
\providecommand{\dec}[2]{\ensuremath{\overset{#1}{\underset{#2}{\gtrless}}}}
\newcommand{\myvec}[1]{\ensuremath{\begin{pmatrix}#1\end{pmatrix}}}
\newcommand{\mydet}[1]{\ensuremath{\begin{vmatrix}#1\end{vmatrix}}}
\numberwithin{equation}{subsection}
\makeatletter
\@addtoreset{figure}{problem}
\makeatother
\let\StandardTheFigure\thefigure
\let\vec\mathbf
\renewcommand{\thefigure}{\theproblem}
\def\putbox#1#2#3{\makebox[0in][l]{\makebox[#1][l]{}\raisebox{\baselineskip}[0in][0in]{\raisebox{#2}[0in][0in]{#3}}}}
     \def\rightbox#1{\makebox[0in][r]{#1}}
     \def\centbox#1{\makebox[0in]{#1}}
     \def\topbox#1{\raisebox{-\baselineskip}[0in][0in]{#1}}
     \def\midbox#1{\raisebox{-0.5\baselineskip}[0in][0in]{#1}}
\vspace{3cm}
\title{ASSIGNMENT 4}
\author{C.RAMYA TULASI}
\maketitle
\newpage
\bigskip
\renewcommand{\thefigure}{\theenumi}
\renewcommand{\thetable}{\theenumi}
Download all python codes from 
\begin{lstlisting}
https://github.com/CRAMYATULASI/ASSIGNMENT4/tree/main/ASSIGNMENT4/CODES
\end{lstlisting}
%
and latex-tikz codes from 
%
\begin{lstlisting}
https://github.com/CRAMYATULASI/ASSIGNMENT4/tree/main/ASSIGNMENT4
\end{lstlisting}
%
\section{Question No 2.6}
Find out whether the lines representing the following pairs of linear equations intersect at a point are parallel or coincident.
%
\begin{enumerate}
%\begin{multicols}{2}
\item
\begin{align}
\begin{split}
\myvec{5 & -4 }\vec{x}&=-8
\\
\myvec{7 & 6 }\vec{x}&=9 \label{1.0.1}
\end{split}
\end{align}
\item
\begin{align}
\begin{split}
\myvec{9 & 3 }\vec{x}&=-12
\\
\myvec{18 & 6 }\vec{x}&=-24 \label{1.0.2}
\end{split}
\end{align}
\item
\begin{align}
\begin{split}
\myvec{6 & -3 }\vec{x}&=-10
\\
\myvec{2 & -1 }\vec{x}&=-9\label{1.0.3}
\end{split}
\end{align}
%\end{multicols}
\end{enumerate}
%
\section{SOLUTION}  
\begin{enumerate}
\item
\begin{align}
\begin{split}
\myvec{5 & -4 }\vec{x}&=-8
\\
\myvec{7 & 6 }\vec{x}&=9
\end{split}
\end{align}
The above equations can be expressed as the matrix equation
\begin{align}
\myvec{5 & -4\\7 & 6} \vec{x} = \myvec{-8\\9}
\end{align}
%
The augmented matrix for the above equation is row reduced as follows
\begin{align}
\myvec{5 & -4 & -8\\7 & 6 & 9} 
\xleftrightarrow {R_1\leftarrow \ 7\frac{R_1}{5}}\myvec{7 & \frac{-28}{5} & \frac{-56}{5}\\7 & 6 & 9} 
\\
%\myvec{7 & \frac{-28}{5} &\frac{-56}{5} \\0 & \frac{58}{5} & \frac{101}{5}} 
\xleftrightarrow {R_2\leftarrow R_2 - R_1}\myvec{7 & \frac{-28}{5} &\frac{-56}{5} \\0 & \frac{58}{5} & \frac{101}{5}}
\\
%\myvec{5 & -4 & -8 \\0 & \frac{58}{5} & \frac{101}{5}} 
\xleftrightarrow {R_1\leftarrow 5\frac{R_1}{7}}\myvec{5 & -4 & -8 \\0 & \frac{58}{5} & \frac{101}{5}}
\\
%\myvec{5 & -4 & -8 \\0 & 58 & 101} 
\xleftrightarrow {R_2\leftarrow 5R_2}\myvec{5 & -4 & -8 \\0 & 58 & 101 }
\end{align}
%
$\because$ row reduction of the $2\times 3$ matrix
%
\begin{align}
\myvec{5 & -4 & -8\\7 & 6 & 9} \label{2.0.7}
\end{align}
%
results in a matrix with 2 nonzero rows, its rank is 2. 
%
Similarly, the rank of the matrix 
\begin{align}
\myvec{5 & -4 \\7 & 6 } \label{2.0.8}
\end{align}
%
is also 2.
%
\begin{align}
\because Rank \myvec{5 & -4\\7 & 6} &= Rank\myvec{5 & -4 & -8\\7 & 6 & 9} \nonumber\\
 &=dim \myvec{5 & -4\\7 & 6}\nonumber\\
 &=2
\end{align}

$\therefore$ Given lines \eqref{1.0.1} have unique solution so we can say they intersect.

\item
\begin{align}
\begin{split}
\myvec{9 & 3 }\vec{x}&=-12
\\
\myvec{18 & 6 }\vec{x}&=-24
\end{split}
\end{align}

The above equations can be expressed as the matrix equation
\begin{align}
\myvec{9 & 3\\18 & 6} \vec{x} = \myvec{-12\\-24}
\end{align}
%
The augmented matrix for the above equation is row reduced as follows
\begin{align}
\myvec{9 & 3 & -12\\18 & 6 & -24 } 
\xleftrightarrow {R_1\leftarrow \ 7\frac{R_1}{5}}\myvec{7 & \frac{-28}{5} & \frac{-56}{5}\\7 & 6 & 9} 
\\
%\myvec{18 & 6 & -24} \\18 & 6 & -24 
\xleftrightarrow {R_1\leftarrow \frac{18R_1}{9}}\myvec{18 & 6 & -24 \\18 & 6 & -24}
\\
%\myvec{18 & 6 & -24 \\0 & 0 & 0} 
\xleftrightarrow {R_2\leftarrow R_2-R_1}\myvec{18 & 6 & -24 \\0 & 0 & 0}
\\
\end{align}
%
$\because$ row reduction of the $2\times 3$ matrix
%
\begin{align}
\myvec{9 & 3 & -12\\18 & 6 & -24}
\end{align}
%
results in a matrix with 1 nonzero rows, its rank is 1. 
%
Similarly, the rank of the matrix 
\begin{align}
\myvec{9 & 3 \\18 & 6 } 
\end{align}
%
is also 1.
%
\begin{align}
\because Rank \myvec{9 & 3\\18 & 6} &= Rank\myvec{9 & 3 & -12\\18 & 6 & -24}=1\nonumber\\
&< dim\myvec{9 & 3\\18 & 6}=2
\end{align}

$\therefore$ Given lines \eqref{1.0.2} have infinitely many solutions so we can say they coincide.


\item
\begin{align}
\begin{split}
\myvec{6 & -3 }\vec{x}&=-10
\\
\myvec{2 & -1 }\vec{x}&=-9
\end{split}
\end{align}

The above equations can be expressed as the matrix equation
\begin{align}
\myvec{6 & -3\\2 & -1} \vec{x} = \myvec{-10\\-9}
\end{align}
%
The augmented matrix for the above equation is row reduced as follows
\begin{align}
\myvec{6 & -3 & -10\\2 & -1 & -9 } 
\xleftrightarrow {R_1\leftarrow \ \frac{2R_1}{6}}\myvec{2 & -1 & \frac{-10}{3}\\2 & -1 & -9} 
\\
%\myvec{2 & -1 & \frac{-10}{3} \\0 & 0 & \frac{-17}{3} 
\xleftrightarrow {R_2\leftarrow R_2-R_1}\myvec{2 & -1 & \frac{-10}{3} \\0 & 0 & \frac{-17}{3} }
\\
%\myvec{6 & -3 & -10 \\0 & 0 & \frac{-17}{3}} 
\xleftrightarrow {R_1\leftarrow 3R_1}\myvec{6 & -3 & -10 \\0 & 0 & \frac{-17}{3}} 
\\
\end{align}
%
$\because$ row reduction of the $2\times 3$ matrix
%
\begin{align}
\myvec{6 & -3 & -10\\2 & -1 & -9}
\end{align}
%
results in a matrix with 2 nonzero rows, its rank is 2. 
%
Similarly, the rank of the matrix 
\begin{align}
\myvec{6 & -3 \\2 & -1 } 
\end{align}
%
is 1.
%
\begin{align}
\because Rank \myvec{6 & -3 \\2 & -1 } \ne Rank \myvec{6 & -3 & -10\\2 & -1 & -9}
\end{align}
$\therefore$ Given lines \eqref{1.0.3} have no solution so we can say they are parallel.

PLOT OF GIVEN LINES -

\numberwithin{figure}{section}
\begin{figure}[ht!]
Plot of \eqref{1.0.1} -
    \centering
    \includegraphics[width=\columnwidth]{INTERSECTING_LINES.png}
    \caption{INTERSECTING LINES.}
    \label{fig:INTERSECTING LINES.}
\end{figure} 

\numberwithin{figure}{section}
\begin{figure}[ht!]
Plot of \eqref{1.0.2} -
    \centering
    \includegraphics[width=\columnwidth]{SAME_LINES.png}
    \caption{SAME LINES}
    \label{fig:SAME LINES.}
\end{figure} 

\numberwithin{figure}{section}
\begin{figure}[ht]
Plot of \eqref{1.0.3} 
    \centering
   \includegraphics[width=\columnwidth]{PARALLEL_LINES.png}
    \caption{Parallel lines}
    \label{fig: PARALLEL lines.}
\end{figure}    
\end{enumerate}
\end{document}